\documentclass[12pt]{article}
\usepackage{}
%%%%%%%%%%%%%%%%%%%%%%%%
% PREAMBLE
\usepackage{times}
\usepackage{courier}
\usepackage{graphicx}
\graphicspath{ {images/} }
\usepackage{setspace}
\usepackage{textcomp}
\usepackage{listings}
\usepackage[letterpaper, margin=1in]{geometry}
\addtolength{\topmargin}{-0.5in}

\lstset{language=C++, upquote=true, breaklines=true, showstringspaces=false}

\lstdefinelanguage{scad}
{
  morecomment=[l][//],
  commentstyle={\it},
  basicstyle={\small\ttfamily},
  morekeywords={true, false},
  sensitive=true,
  morestring=[b]''
  breaklines=true,
  showstringspaces=false,
}

\begin{document}
\begin{titlepage}
\title{Constructive Solid Geometry for Interactive Environments}
\author{Nicholas Czaban}
\date{\today}
\maketitle
\begin{abstract}
  One barrier to high-quality computer generated images is the exorbitant number of polygons and points needed to define a detailed object. Defining each vertex by hand would take a user hours for even the simplest shapes. One way to streamline the process is to use Constructive Solid Geometry (CSG) to define shapes as combinations of geometric primitives. This paper outlines the advantages and disadvantages of this method, as well as two major algorithms used to render CSG images. Furthermore, an implementation of CSG shapes using OpenGL is described. Although adequate for the purpose of th ecreated software, there is still room for improvement in CSG usage and implementation.
\end{abstract}
\end{titlepage}
\begin{onehalfspace}
\tableofcontents
\listoffigures
\newpage
\end{onehalfspace}
\begin{doublespace}
\section{Constructive Solid Geometry}
\subsection{General Concepts}
Constructive solid geometry is a method for creating more complicated shapes by performing boolean operations on simpler shapes. The available boolean operations include union, intersection, and difference. The types of shapes vary from one implementation to another, but generally include spheres, cubes, and cylinders, as well as their standard transformations (rotations, scaling, translations) \cite{cs_dictionary}. This technique provides a simpler way to create three-dimensional shapes without the need to define each individual vertex or face. CSG objects are organized in binary trees, where leaf nodes are the solid primitives and interior nodes are the boolean operations. The object is then created recursively from the leaves up, with each boolean operation being replaced by the object which it created\cite{gold}.

\begin{figure}[h]
  \includegraphics[width=16cm]{CSG}
  \centering
  \caption{The three types of CSG operations perfomed on a cube and a sphere: (from left) Union, Intersection, Difference}
\end{figure}
\subsubsection{Real-Time Rendering and Image-Based CSG}
One substantial category in CSG is the difference between Real-Time Rendering of CSG objects and Image-Based CSG objects. With real-time rendering, the entirety of the object is rendered, while image-basedd techniques instead pass a two-dimensional image of the CSG shape from a particular angle. Real-time rendering poses a number of issues, particularly for shapes that are interactively modified during runtime. While real-time rendering can be accomplished by passing the shape as a mathematically calculated model, this solution is more applicable to static CSG shapes and does not work well for interactive modifications \cite{open_csg}. The alternative solution, image-based rendering algorithms, avoid most of the complex boundary calculations of constructive solid geometry by storing only an image of the shape in the frame buffer \cite{open_csg}.
\subsection{Use in Computer Graphics}
\subsubsection{Interactive Modification}
In practice, CSG objects are typically rendered with either raycasting or boundary representations, both of which are fairly slow. Boundary representations in particular take a significant amount of time to generate, so the final objects are typically treated as static objects and are not modified \cite{advanced_opengl}. For instance, a scene where a cylindrical hole is slowly drilled in another shape would need to continually add more polygons to show the growing hole. Calculating the boundaries for each iteration of the shape would be inordinately complex. A raycasting method would be able to handle this scenario at the same efficiency as any other scenario, but the algorithm itself is still too slow for real-time rendering.
\subsubsection{Raytracing}
In comparison with raycasting, which determines only the first object struck by a light ray, raytracing tracks all objects struck by a light ray, making it far more useful for interactive environments. When using raytracing algorithms for CSG, the boolean operations can be reduced to one-dimensional subproblems \cite{raytrace}. For example, a ray passing through two objects along its path has a line segments representing when the ray entered and exited each primitive. These line segments can then be subjected to the boolean operations. The resulting line segment will be equivalent to a line segment generated from a ray passing through the complete CSG primitive. Although individual ray calculations are fast, the excessive number of rays needed to accurately reproduce the model can make more complicated scenes unsuitable for real-time rendering \cite{interactive_csg}.
\subsection{CSG Algorithms}
Given the rendering complexity of constructive solid geometry, several algorithms have been developed to improve the speed and quality of CSG primitives. Two such algorithms are the Goldfeather Algorithm and the layered Goldfeather algorithm \cite{hardware_csg}.
\subsubsection{Goldfeather Algorithm}
Goldfeather's algorithm quickly became the precursor to all image-based CSG algorithms \cite{hardware_csg}. The algorithm, originally designed for the Pixel-Planes graphics hardware, uses partial products to allow effective use of the z-buffer on supported graphics hardware. The algorithm is of two parts: the passed CSG tree is first restructured into an equivalent tree more easily used by the second portion of the algorithm, which traverses the new tree and returns quadratic coefficients to be read by Pixel-Powers \cite{gold}. Pixel-Powers, an extension of Pixel-Planes, calculates the values of a quadratic function $Ax^2 + Bxy + Cy^2 + Dx + Ey + F$, where $x$ and $y$ are screen coordinates and $A$ through $F$ are constants. This quadratic equation is used to define the boundary surfaces of all primitive solids in the implementation, including cones, cylinders, and ellipsoids \cite{gold}.\\

When perfoming CSG unions in this algorithm, the visible section of the first object is stored in a z-buffer, denoted in Goldfeather's paper as \texttt{ZMIN}, and the visible section of the second object is stored in another z-buffer, denoted \texttt{ZTEMP}. For all the pixels in the visible area of the second object, a flag \texttt{F1} is set to true. Then, for any pixels where \texttt{F1} is set, if \texttt{ZTEMP} greater than \texttt{ZMIN} (i.e. the second object is deeper in model space than the first object), the \texttt{F1} flag is cleared. The remaining pixels with \texttt{F1} set represent areas where the second object is visible from behind the first object. Both objects are then drawn without any overlap.\\

In difference operations, the flag \texttt{F1} is set for all pixels of the first object and the front faces are stored in \texttt{ZTEMP}. If the \texttt{ZTEMP} values lie between the front and back of the second object, \texttt{F1} is cleared. The remaining \texttt{F1} pixels are not within the second object, but are on the front of the first object; the corresponding values of \texttt{ZTEMP} are passed into \texttt{ZMIN}.\\

For intersection problems, \texttt{F1} is set for the front of the first object, then cleared if the value of \texttt{ZTEMP} is not between the front and back of the second object. Then, the same comparison is made with the second object stored in \texttt{ZTEMP} and compared between the front and back faces of the first object. Only the pixels with the flag set for both sets are drawn.
\subsubsection{Layered Goldfeather Algorithm}
In their extension of Goldfeather's algorithm, Stewart et. al. exploit instances where the maximum depth complexity $k$ (the number of primitives covering each pixel) is less than the number of primitives $n$ in a normalized CSG tree \cite{layered_gold}. For example, a scene with five spheres in a line has no overlapping objects when the viewer is orthogonal to the line. In this case, $k=1$ and is less than the number of primitives, $n=5$. If the viewer looked directly down the line, the depth complexity equals the number of primitives and is not suitable for the layered algorithm.\\

The layered algorithm defines layers by their surface parity, or the number of edges in the z-buffer. Parity is stored as a flag in the algorithm; parity is even when outside an object and odd when inside an object. While generally effective, this definition places important constraints on the usable surfaces: the surface cannot self-intersect, and all volumes must be completely covered in a surface with no holes \cite{layered_gold}.\\

The original Goldfeather algorithm performs a single clipping operation in $an+b$ time, where $a$ and $b$ are constants determined by polygon rasterization speed and z-buffer copying, respectively \cite{layered_gold}. Each primitive is clipped once, giving an upper runtime bound of $O(n^2)$. In the layered Goldfeather algorithm, each primitive is rendered once into the layer and once to clip the layer. Each layer is copied into the z-buffer, so the time to clip a single layer in the layered approach is $2an+b$. In total, with each layer clipped once, the layered algorithm runs in $2akn+bk$, or $O(kn)$ \cite{layered_gold}. As stated above, the algorithm performs better than the original Goldfeather method when $k<n$.

\subsubsection{Other Techniques}
In addition to the core algorithms, there are other, more minute techniques that can improve performance when rendering CSG objects. In Nicholas Wilt's implementation of CSG in the Object Oriented Ray Tracing (OORT) class library, complicated calculations can be short-circuited by creating optimized CSG trees where simpler objects are the left children of each boolean operation. In OORT, ray tracing calculations between two objects on a CSG tree always test against the left object first. Thus, functions like \texttt{CSGIntersection} can avoid testing rays against more complex shapes if the left shape returns FALSE \cite{raytrace}.\\

The Goldfeather algorithm uses two separate z-buffers, which can create issues on systems that do not support multiple z-buffers. Alternative approaches, such as copying one set of z-values into main memory, are inadequately slow and values can lose exactness during copying. One workaround is to save the second z-buffer into another, unused buffer, such as the color buffer. Commonly referred to as render-to-texture, color data is discarded and the texels of the new p-buffer have a one-to-one match to the pixels in the frame buffer \cite{hardware_csg}.\\

In the layered Goldfeather approach, the biggest detriment to its speed is reading from the stencil buffer to calculate the maximum depth complexity. However, with modern graphics hardware the depth complexity can be calculated without reading directly from the stencil buffer by using extensions built in to NVidia and ATI hardware, such as NV\_occlusion\_query \cite{hardware_csg}.\\

Another performance-increasing method is to normalize the binary tree, which pushes union operations towards the root of the tree and difference/intersection operations towards the leaves \cite{advanced_opengl}. Although normalizing the tree can increase the height and number of primitives in the tree, overall the normal form can be processed more quickly. In the normalized tree, no intersection or difference node has a union operator in its left subtree, and a single primitive as its right child. As union operations, as opposed to difference and intersection, are additive, placing these nodes nearer to the root of the tree saves time, since the unimportant sections are removed earlier in the rendering process. To normalize the tree, several set properties are used recursively on the tree; $X-(Y\cup Z)=(X-Y)-Z$ and $X-(Y\cap Z) = (X-Y)\cup(X-Z)$, for example. Once the normal tree is created, extraneous nodes can be pruned; intersections and differences where the two objects do not touch are deleted or replaced with the left child, respectively.\\

A more recent technique involves partitioning the scene into voxels (small cubes) instead of pixels using a method similar to Goldfeather's algorithm above, where each voxel of the scene has a flag determining whether it is inside or outside an object \cite{geoinformation}. There are two major variants on the voxelization process: voxelizing and combining each primitive, or voxelizing along rays. While the first method requires more memory, it is advantageous if the user requires the voxelized version of each subtree. Once this preprocessing phase is completed, inclusion tests can be performed in constant time. However, the voxelization method is still unsuitable for real-time use.\\

Many of the other algorithms that are developed for CSG are either unsuitable for the hardware of graphical processing units (GPUs). In recent years, the GPU memory capacity has grown at a slower rate than the computational power of the GPUs themselves, creating a bottleneck for algorithms that rely on the GPU \cite{interactive_csg}. For example, raytracing algorithms for CSG divide the rendered object into a set of intervals between when the primitives are intersected, but GPUs cannot store all the intervals for thousands of individual rays unless the scene is relatively simple. Another approach relies on recursively raytracing the left and right subtrees, recording where the ray enters, exits, and misses an object. Again, this solution is not ideal for a GPU because of the memory needed to store the three states of the ray and the recursive nature of the algorithm \cite{interactive_csg}.
\section{Software}
\subsection{Display Program}
The first implementation of constructive solid geometry is a simple display program. It takes as a command-line argument an .stl file, containing the vertices of the shape to be rendered in OpenGL. The shape can be rotated on its axis by clicking and dragging the shape.
\subsubsection{CSG Usage}
The shapes are individually created using OpenSCAD, a program for computer assisted design. The shapes are exported as .stl files, which can be read using the reader function located in \texttt{stlReader.cpp}. An example of a typical .scad file is shown in Figure 2.\\
\lstset{language=scad}
\begin{figure}[h]
\begin{lstlisting}
  intersection() {
    cube(15, center=true);
    rotate(90, [1,0,0]) sphere(r=10, \$fn=50);
   }
\end{lstlisting}
\centering
\caption{An exampe code segment from a .scad file}
\end{figure}

The \texttt{intersection()} function takes no inputs in its declaration, instead operating on the objects created within its code block; \texttt{union()} and \texttt{difference()} work in the same way. The \texttt{cube()} function can either create a regular cube when passed a single number, as in Figure 2, or create rectangular prisms with user-defined dimensions. Instead of a single number, the user  passes the dimensions as such: \texttt{cube([15,10,12]);}. The \texttt{center=true} parameter sets the cube with its center at the origin; a parameter of \texttt{center=false} draws the cube with its bottom corner at the origin. Cylinders have independently defined radii for both ends; a cone can be created by setting one radius to 0. As seen in the \texttt{rotate} call, vectors are contained within square brackets. For the sake of simplicity, and to better explore the possibilities of CSG, only rectangular prisms, cones, cylinders, and spheres are used to build each object. For transformations, only rotate and translate are used.\\

In order to manipulate the objects created with OpenSCAD, the CSG objects were exported in .stl files. Each file is formatted as shown in Figure 3:
\begin{figure}[h]
\begin{lstlisting}
  facet normal 0.603056 0.409837 0.684366
    outer loop
      vertex 3.33769 2.42497 3.63721
      vertex 3.14634 2.28595 3.88909
      vertex 3.48337 2.21061 3.63721
    endloop
  endfacet
\end{lstlisting}
\centering
\caption{A single triangle as saved in an ASCII .stl file}
\end{figure}

The object is broken down into a series of triangles and a normal vector is calculated. Using this pattern, the .stl files are read by the helper program, stlReader.cpp. stlReader.cpp uses the struct \texttt{Facet} to store the data for each triangle, with each \texttt{Facet} being added to a vector containing all the triangles of the shape. The helper program is currently only configured to read .stl files that use the ASCII-style layout, and cannot read binary .stl files.\\

In total, seven objects are created using OpenSCAD. These include, but are not limited to: a die, a torus, an asteroid, a spaceship, and a 6-sided cookie-cutter.
\begin{figure}[h]
  \includegraphics[width=9cm]{Cookie}
  \centering
  \caption{A 6-sided cookie cutter as seen in the OpenSCAD environment. The shapes include the four card suits (heart, club, diamond, spade), a cross, and a star.}
\end{figure}
\subsubsection{OpenGL Implementation}
Once the .stl file is parsed, the vector of \texttt{Facet}s is fed into OpenGL's \texttt{GL{\bf\_}TRIANGLES} primitive to rebuild the object in a more interactive environment. The display program uses a white specular point light and a material shininess of 50. A \texttt{mouse} and \texttt{motion} function enable the user to rotate the object on its axis, with the rotations being performed within the \texttt{display} function; the user can also quit the program with the 'q' key. Within the \texttt{glBegin} and \texttt{glEnd} block, the vector \texttt{newShape} is scanned, with each \texttt{Facet}'s normal vector and vertices passed to the respective OpenGL command. In order to correctly light the triangle elements, the vertices are passed in the reverse order of how they are printed in the .stl file.
\subsection{Asteroid Game}
\begin{figure}
  \includegraphics[width=11cm]{Game}
  \centering
  \caption{A screenshot from the asteroid game.}
\end{figure}
The second implementation of constructive solid geometry is an interactive game, where the user controls a spaceship and avoids asteroids. In comparison to the display program, this program enables the \texttt{GL{\bf\_}COLOR{\bf\_}MATERIAL} state in its \texttt{init} call, which allows the \texttt{glColor} function to be used in conjunction with lighting. The spaceship surface is colored a light grey, with the engine trails colored blue; the asteroids are colored red. For the engine color, the code checks to see if the normal vector is pointing in the same direction as the engines, as well as if the z-value of the triangle's first vertex is at -1.5, at the back of the ship. The game removes the mouse-controlled camera in place of a camera set with the \texttt{gluLookAt} function to constantly be directly behind the spaceship. Additionally, more key bindings are added to control the movement of the spaceship. Each of the WASD keys has a respective boolean variable that is set to true when the key is pressed and false when the key is released. This method allows for diagonal movement as well as cardinal movement. Although the display program used \texttt{glOrtho} to set up the projection, the game uses \texttt{gluProjection}. This change was needed in order to simulate the correct depth perception on the obstacles.\\

In the idle loop of the game, the position values of the spaceship are changed. If one of the directional booleans is true, \texttt{xpos} or \texttt{ypos} is changed accordingly. When the scene is redrawn, the camera position uses \texttt{xpos} and \texttt{ypos} to remain behind the ship. Another variable, \texttt{zpos} is also changed; this variable does not translate the spaceship itself, but instead translates all other obstacles onscreen. The \texttt{zpos} variable is cyclical; once the value is less than -100, \texttt{zpos} is reset to The ``playing area'' of the game ends when \texttt{zpos=-100}, at which point it resets to \texttt{zpos = 50*speed+25}, and the \texttt{speed}, \texttt{zpos}'s rate of change, is increased by 0.2 to a max speed of 1.\\

The increased movement of the game objects highlighted one major challenge of converting the OpenSCAD designs to OpenGL. OpenSCAD uses a right-handed coordinate system, while OpenGL uses a left-handed system.
\begin{figure}[h]
  \includegraphics[width=12cm]{Right-Hand}
  \centering
  \caption{The OpenSCAD viewer. As indicated by the legend in the bottom-left corner, the program uses a right-handed coordinate system (the positive z-axis is pointing up). In a left-handed coordinate system, such as the one used by OpenGL, the negative z-axis would be pointing up.}
\end{figure}
As the objects moved towards and away from the camera, parts of the objects that should have been hidden were visible, and vice versa. For example, the spaceship engines jut out from the back of the ship. Thus, a viewer positioned in between the two engines should be able to see the right curve of the left engine and the left curve of the right engine, as shown in Figure 7. However, the viewer instead could see the left curve of the left engine and the right curve of the right engine, as seen below right; this perspective made the engines appear recessed instead of external. In order to solve this issue, the z-coordinate of each vertex was multiplied by -1. In conjunction with this fix, the x-value of each obstacle's normal vector was multiplied by -1. This change produced better lighting effects on the obstacles with the new vertex values. The normal vector for the spaceship was left as is.
\begin{figure}[h]
  \includegraphics[width=8cm]{GoodPersp}
  \includegraphics[width=8cm]{BadPersp}
  \centering
  \caption{Two different views of the spaceship model. The left model is the intended view, the right model is how the object was initially rendered in OpenGL.}
\end{figure}

Another struct, \texttt{GameObject}s, is used to hold two three-element arrays; these three variables determine the rotation and position of each obstacle onscreen. A vector of these \texttt{GameObject}s is created in the \texttt{init} function and populated with 7 asteroids. In a for loop, each obstacle in the vector is drawn with its corresponding rotation and position. For the \texttt{glTranslate} calls, the following equation is used to center the asteroids in the xy-plane of the game screen: \texttt{30*obstacles[i].position[0]-15} for x and \texttt{30*obstacles[i].position[1]-15}for y. The bounds on the spaceship are between -10 and 10 on both the x- and y-axis, and this formula bounds the position of the asteroid between -15 and 15. The z-position of the asteroid is determined by \texttt{-90*obstacles[i].position[2]-zpos}.\\

The \texttt{glRotate} calls use the random values with the equation \texttt{2*obstacles[i].rotation[0]-1} for x, \texttt{2*obstacles[i].rotation[1]-1}, for y, and \texttt{2*obstacles[i].rotation[2]-1} for the z-value of the rotation vector. The angle of rotation is calculated by \texttt{-1.5*zpos}, causing the asteroid to continually rotate as it moves through the scene. By multiplying by 2 and subtracting 1, each component of the rotation vector is bounded between -1 and 1.
\section{Conclusion}
As a whole, the field of constructive solid geometry poses a number of difficult problems for computer scientists. With real-time rendering infeasible, and the alternative algorithms only suitable in a handful of cases, there is still much research to be done. However, the existing solutions are more than adequate to produce usable and complete software, as evidenced by the display program and the asteroid game. Although the OpenGL implementation was not ideal for rendering the 
\newpage
\end{doublespace}
\begin{onehalfspace}
\section{Source Code}
\subsection{OpenSCAD}
\subsubsection{die.scad}
\lstinputlisting[language=scad]{die.scad}
\subsubsection{connector.scad}
\lstinputlisting[language=scad]{connector.scad}
\subsubsection{torus.scad}
\lstinputlisting[language=scad]{torus.scad}
\subsubsection{asteroid.scad}
\lstinputlisting[language=scad]{asteroid.scad}
\subsubsection{pipes.scad}
\lstinputlisting[language=scad]{pipes.scad}
\subsubsection{spaceship.scad}
\lstinputlisting[language=scad]{spaceship.scad}
\subsubsection{cookie\_cutter.scad}
\lstinputlisting[language=scad]{cookie_cutter.scad}

\subsection{C++}
\subsubsection{stlReader.cpp}
\lstinputlisting[language=C++]{stlReader.cpp}
\subsubsection{display.cpp}
\lstinputlisting[language=C++]{display.cpp}
\subsubsection{game.cpp}
\lstinputlisting[language=C++]{game.cpp}

\newpage
\begin{thebibliography}{10}
\bibitem{cs_dictionary}
  Butterfield, Andrew and Gerard Ngondi,
  \textit{A Dictionary of Computer Science},
  Oxford University Press,
  7th edition,
  2016

\bibitem{gold}
  Goldfeather, Jack, Jeff Hultquist, and Henry Fuchs. ``Fast Constructive Solid Geometry in the Pixel-Powers Graphics System.'' {\it Computer Graphics, SIGGRAPH '86 Proceedings}, vol. 20, no. 4, 1986,
pp. 107-116
  
\bibitem{hardware_csg}
  Kirsch, Florian and J\"{u}rgen D\"{o}llner. ``Rendering Techniques for Hardware-Accelerated Image-Based CSG.'' {\it Journal of WSCG}, vol. 12, no. 2, 2004, pp. 221-228.

\bibitem{open_csg}
  Kirsch, Florian and J\"{u}rgen D\"{o}llner. ``OpenCSG: A Library for Image-Based CSG Rendering.'' {\it Proceedings of the FREENIX/ Open Source Track, 2005 USENIX Annual Technical Conference}. 2005.

\bibitem{advanced_opengl}
  McReynolds, Tom and David Blythe. {\it Advanced Graphic Programming Using OpenGL}. Morgan Kaufmann, San Francisco, 2005.

\bibitem{geoinformation}
  Ooms, Kristien. ``A 3D inclusion test on large dataset.'' {\it Developments in 3D Geo-Information Sciences}. Edited by Philippe De Maeyer and Tijs Neutens. Springer, New York, 2010.

\bibitem{layered_gold}
  Stewart, Nigel, Geoff Leach, and Sabu John. ``An Improved Z-Buffer CSG Rendering Algorithm.'' {\it 1998 Eurographics / SIGGRAPH Workshop on Graphics Hardware}, ACM, 25-30, 1998.

\bibitem{raytrace}
  Wilt, Nicholas. {\it Object-Oriented Ray Tracing in C++}. John Wiley \& Sons, New York, 1994, pp 207-227.

\bibitem{interactive_csg}
  Ulyanov, D., D. Bogolepov, and V. Turlapov. ``Interactive Visualization of Constructive Solid Geometry Scenes on Graphic Processors.'' {\it Programming and Computer Software}, vol. 43, no. 4, 2017, pp. 258-267.
\end{thebibliography}
\end{onehalfspace}
\end{document}
