\documentclass[12pt]{article}
\usepackage{}
%%%%%%%%%%%%%%%%%%%%%%%%
% PREAMBLE
\usepackage{times}
\usepackage{courier}
\usepackage{setspace}
\usepackage{textcomp}
\usepackage{listings}
\linespread{1.25}
\usepackage[letterpaper, margin=1in]{geometry}
\addtolength{\topmargin}{-0.5in}

\lstset{language=C++, upquote=true}

\lstdefinelanguage{scad}
{
  morekeywords={true, false},
  sensitive=true,
  commentstyle=\it,
  morecomment=[l][//],
  morestring=[b]''
  breaklines=true,
  showstringspaces=false,
}

\begin{document}
\title{Constructive Solid Geometry for Interactive Environments}
\author{Nicholas Czaban}
\date{\today}
\maketitle

\section{Constructive Solid Geometry}
\subsection{General Concepts}
\indent Constructive Solid Geometry (CSG) is a method for creating more complicated shapes by performing boolean operations on simpler shapes. The available boolean operations include unions, intersections, and differences; the types of shapes vary from one implementation to another, but generally include spheres, cubes, and cylinders, as well as their standard transformations (rotations, scaling, translations)\cite{cs_dictionary}. This technique provides a simpler way to create three-dimensional shapes without the need to define each vertex
\subsubsection{Real-Time Rendering and Image-Based CSG}
\indent One substantial category in CSG is the difference between Real-Time Rendering of CSG objects and Image-Based CSG objects. Real-time rendering poses a number of issues, particularly for shapes that are interactively modified during runtime. While real-time rendering can be accomplished by passing the shape as a mathematically calculated model, this solution is more applicable to static CSG shapes and does not work well for interactive modifications\cite{open_csg}. The alternative solution, image-based rendering algorithms, avoid most of the complex boundary calculations of constructive solid geometry by storing only an image of the shape in the frame buffer\cite{open_csg}. 
\subsection{Use in Computer Graphics}
\subsubsection{Interactive Modification}
\subsubsection{Ray Tracing}
\subsection{CSG Algorithms}
\indent Given the rendering complexity of constructive solid geometry, several techniques have been developed to improve the speed and quality of CSG primitives. In Nicholas Wilt's implementation of CSG in the Object Oriented Ray Tracing (OORT) class library, complicated calculations can be short-circuited by creating optimized CSG trees where simpler objects are the left children of each boolean operation. In OORT, ray tracing calculations between two objects on a CSG tree always test against the left object first. Thus, functions like \texttt{CSGIntersection} can avoid testing rays against more complex shapes if the left shape returns FALSE\cite{raytrace}.
\section{Software}
\subsection{Display Program}

\subsection{Asteroids}
\indent 
\subsubsection{CSG Usage}
\indent The shapes are individually created using OpenSCAD, a program for computer assisted design. The shapes are exported as .stl files, which can be read using the reader function located in \texttt{stlReader.cpp}. The code within a standard .scad file will look similar to this:
\lstset{language=scad}
\begin{lstlisting}
  // A 6-sided die with rounded corners and circular faces
  intersection() {
    cube(15, center=true);
    sphere(r=10, \$fn=50);
  }
  
\end{lstlisting}
\subsubsection{OpenGL Implementation}
\section{Conclusion}
\newpage
\section{Appendix}
\subsection{stlReader.cpp}
\lstinputlisting[language=C++]{stlReader.cpp}
\newpage
\begin{thebibliography}{10}
\bibitem{cs_dictionary}
  Andrew Butterfield and Gerard Ngondi,
  \textit{A Dictionary of Computer Science},
  Oxford University Press,
  7th edition,
  2016

\bibitem{hardware_csg}
  Kirsch, Florian and J\"{u}rgen D\"{o}llner. ``Rendering Techniques for Hardware-Accelerated Image-Based CSG.'' {\it Journal of WSCG}, vol. 12, no. 2, 2004, pp. 221-228.

\bibitem{open_csg}
  Kirsch, Florian and J\"{u}rgen D\"{o}llner. ``OpenCSG: A Library for Image-Based CSG Rendering.'' {\it Proceedings of the FREENIX/ Open Source Track, 2005 USENIX Annual Technical Conference}. 2005.

\bibitem{advanced_opengl}
  McReynolds, Tom and David Blythe. {\it Advanced Graphic Programming Using OpenGL}. Morgan Kaufmann, San Francisco, 2005.

\bibitem{geoinformation}
  Ooms, Kristien. ``A 3D inclusion test on large dataset.'' {\it Developments in 3D Geo-Information Sciences}. Edited by Philippe De Maeyer and Tijs Neutens. Springer, New York, 2010.

\bibitem{raytrace}
  Wilt, Nicholas. {\it Object-Oriented Ray Tracing in C++}. John Wiley \& Sons, New York, 1994, pp 207-227.

\bibitem{interactive_csg}
  Ulyanov, D., D. Bogolepov, and V. Turlapov. ``Interactive Visualization of Constructive Solid Geometry Scenes on Graphic Processors.'' {\it Programming and Computer Software}, vol. 43, no. 4, 2017, pp. 258-267.
\end{thebibliography}
\end{document}
